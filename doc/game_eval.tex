\documentclass[12pt,a4paper]{article}

\usepackage[spanish]{babel}
\usepackage[T1]{fontenc}

\usepackage[top=3cm, bottom=2.5cm, left=2cm, right=2cm]{geometry}

\begin{document}

\section*{Evaluación de Juego}

\begin{center}
  \huge{Juego:
    \fbox{
      \begin{minipage}{0.6\linewidth}
        \hfill\vspace{1cm}
      \end{minipage}
    }
  }
\end{center}
\subsection*{Criterios de evaluación: }

\begin{tabular}{|l|c|c|c|c|c|p{7cm}| }
  \hline
   & 1 & 2 & 3 & 4 & 5 & Comentarios \\
  \hline
  \hline
  Novedad. Factor único & & & & & & \\
   & & & & & & \\
  \hline
  Calidad Gráfica & & & & & & \\
   & & & & & & \\
  \hline
  Calidad Sonora & & & & & & \\
   & & & & & & \\
  \hline
  Facilidad de Aprender & & & & & & \\
   & & & & & & \\
  \hline
  Percepción de Objetivos & & & & & & \\
   & & & & & & \\
  \hline
  Reto Continuo  & & & & & & \\
   & & & & & & \\
  \hline
  Concentración Necesaria & & & & & & \\
   & & & & & & \\
  \hline
  Habilidad vs Suerte & & & & & & \\
   & & & & & & \\
  \hline
  Aleatoriedad, variedad & & & & & & \\
   & & & & & & \\
  \hline
  Factor de Acción & & & & & & \\
   & & & & & & \\
  \hline
  Control del Jugador & & & & & & \\
   & & & & & & \\
  \hline
  Valor por Partida & & & & & & \\
   & & & & & & \\
  \hline
\end{tabular}

\subsection*{Comentarios adicionales:}

\centering\fbox{
	\begin{minipage}{0.85\linewidth}
		\hfill\vspace{6cm}
	\end{minipage}
	}

\newpage

\begin{description}
\item[Novedad. Factor único] \hfill \\
  Originalidad del juego.
\item[Calidad Gráfica] \hfill \\
  Los gráficos no solo son usar la última tecnología sino la adecuación al
  juego.
\item[Calidad Sonora] \hfill \\
  Medir si el audio proporciona algo al juego, esta relacionado.
\item[Facilidad de Aprender] \hfill \\
  ``Fácil de aprender, difícil de dominar''. Juego que puede ser aprendido
  intuitivamente con pocas barreras al aprendizaje.
\item[Percepción de Objetivos] \hfill \\
  Queda claro los objetivos a corto, medio y largo plazo del juego.
\item[Reto Continuo] \hfill \\
  Evitar conseguir logros sin suficiente reto.x
\item[Concentración Necesaria] \hfill \\
  Concentración como medida de la inmersión en el juego.
\item[Habilidad vs Suerte] \hfill \\
  Los jugadores valoran más avanzar por habilidad propia en vez de por suerte.
\item[Aleatoriedad, variedad] \hfill \\
  Aleatoriedad es la ausencia de patrones. Variedad es la existencia de
  elementos diversos para mantener la atención del jugador.
\item[Factor de Acción] \hfill \\
  Acciones por segundo que realiza el jugador. Más no significa mejor; este
  factor tiene que ser adecuado al tipo de juego.
\item[Control del Jugador] \hfill \\
  Un buen control facilita al jugador realizar las acciones necesarias durante
  el juego.
\item[Valor por Partida] \hfill \\
  Satisfacción de una partida completa, que causa que el jugador quiera volver a
  jugar.

\end{description}

\end{document}
